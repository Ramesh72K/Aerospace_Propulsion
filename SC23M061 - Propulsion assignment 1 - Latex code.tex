\documentclass[]{article}

%These tell TeX which packages to use.
\usepackage{array,epsfig}
\usepackage{amsmath}
\usepackage{amsfonts}
\usepackage{amssymb}
\usepackage{amsxtra}
\usepackage{amsthm}
\usepackage{mathrsfs}
\usepackage{color}
\usepackage{graphicx}
\usepackage{xcolor}
\usepackage{subcaption}
\usepackage{matlab-prettifier}
\usepackage{siunitx}
\usepackage{fancyhdr}
\raggedbottom


%Here I define some theorem styles and shortcut commands for symbols I use often
\theoremstyle{definition}
\newtheorem{defn}{Definition}
\newtheorem{thm}{Theorem}
\newtheorem{cor}{Corollary}
\newtheorem*{rmk}{Remark}
\newtheorem{lem}{Lemma}
\newtheorem*{joke}{Joke}
\newtheorem{ex}{Example}
\newtheorem*{soln}{Solution}
\newtheorem{prop}{Proposition}

\newcommand{\lra}{\longrightarrow}
\newcommand{\ra}{\rightarrow}
\newcommand{\surj}{\twoheadrightarrow}
\newcommand{\graph}{\mathrm{graph}}
\newcommand{\bb}[1]{\mathbb{#1}}
\newcommand{\Z}{\bb{Z}}
\newcommand{\Q}{\bb{Q}}
\newcommand{\R}{\bb{R}}
\newcommand{\C}{\bb{C}}
\newcommand{\N}{\bb{N}}
\newcommand{\M}{\mathbf{M}}
\newcommand{\m}{\mathbf{m}}
\newcommand{\MM}{\mathscr{M}}
\newcommand{\HH}{\mathscr{H}}
\newcommand{\Om}{\Omega}
\newcommand{\Ho}{\in\HH(\Om)}
\newcommand{\bd}{\partial}
\newcommand{\del}{\partial}
\newcommand{\bardel}{\overline\partial}
\newcommand{\textdf}[1]{\textbf{\textsf{#1}}\index{#1}}
\newcommand{\img}{\mathrm{img}}
\newcommand{\ip}[2]{\left\langle{#1},{#2}\right\rangle}
\newcommand{\inter}[1]{\mathrm{int}{#1}}
\newcommand{\exter}[1]{\mathrm{ext}{#1}}
\newcommand{\cl}[1]{\mathrm{cl}{#1}}
\newcommand{\ds}{\displaystyle}
\newcommand{\vol}{\mathrm{vol}}
\newcommand{\cnt}{\mathrm{ct}}
\newcommand{\osc}{\mathrm{osc}}
\newcommand{\LL}{\mathbf{L}}
\newcommand{\UU}{\mathbf{U}}
\newcommand{\support}{\mathrm{support}}
\newcommand{\AND}{\;\wedge\;}
\newcommand{\OR}{\;\vee\;}
\newcommand{\Oset}{\varnothing}
\newcommand{\st}{\ni}
\newcommand{\wh}{\widehat}

%Pagination stuff.
\setlength{\topmargin}{-.3 in}
\setlength{\oddsidemargin}{0in}
\setlength{\evensidemargin}{0in}
\setlength{\textheight}{9.in}
\setlength{\textwidth}{6.5in}
\pagestyle{empty}

\pagestyle{fancy}
\fancyhf{} % Clear existing header and footer
\fancyfoot[C]{\thepage} % Centered page number at the bottom
\renewcommand{\headrulewidth}{0pt} % Remove the horizontal line at the top


\begin{document}
\begin{center}
\vspace{.4cm}
{\bf{Aerospace Propulsion - Assignment 1}}
\vspace{.2cm}
\\
{\bf Ramesh M} (SC23M061)
\end{center}

%1st Problem Starts
\begin{enumerate}
\item You are to pick an engine for a transport aircraft flying at Mach 0.8 at sea level on a standard day. The exit burner total temperature is 1667 K, and \( \Delta H = 41.8 \, \text{MJ/kg} \). The air mass flow rate in the core is \( 81.64 \, \text{kg/s} \). Use an ideal cycle analysis.
\begin{enumerate}
\item Find the dimensionless quantity, \( \frac{F}{\dot{m_t}\,a_a}\), and the dimensional quantities \( F\) and \(TSFC\) for the following engines:
\begin{enumerate}
\item Ramjet
\item Turbojet (\( \pi_c = 16 \))
\item Turbojet with afterburner (\( \pi_c = 16 \); total afterburner temperature is \(2333\, K\))
\item Turbofan with exhausted fan (\( \pi_c = 16\), \( \pi_f = 4.0\), \( \alpha = 1\))
\end{enumerate}
\item Which engine will you choose and why?
\end{enumerate}
\emph{\textbf{Solution:}}
\begin{enumerate}
\item
\begin{tabular}{|c|c|c|c|c|}
\hline
S.No & Engine type & TSFC $(\frac{kg}{s.N})$ & Thrust, \( F \) (N) & \( \frac{F}{\dot{m} \cdot a_a} \) \\
\hline
1 & Ramjet & \( 9.3724 \times 10^{-5} \) & \( 2.8105 \times 10^{4} \) & 1.0117 \\
\hline
2 & Turbojet & \( 3.8859 \times 10^{-5} \) & \( 6.7786 \times 10^{4} \) & 2.4402 \\
\hline
3 & Turbojet with afterburner & \( 4.7328 \times 10^{-5} \) & \( 9.9566 \times 10^{4} \) & 3.5842 \\
\hline
4 & Turbofan with exhausted fan & \( 2.2181 \times 10^{-5} \) & \( 8.4005 \times 10^{4} \) & 1.5120 \\
\hline
\end{tabular}
\item
Turbofan with exhausted fan gives comparable thrust with less TSFC. So, we can choose the Turbofan with exhausted fan for these conditions.
\end{enumerate}
%1st Problem ends

%2nd Problem starts
\item The effect of compression ratio on a turbojet engine with an afterburner must be analyzed for the following operating parameters:
$ M_0 = 2.0$, $P_0 = 10\, \text{kPa}$, $T_0 = -45^\circ\, \text{C}$, $\gamma = 1.4$, $c_p = 1004\, \text{J/(kg·K)}$, $\pi_c = 12$, $\tau_\lambda = \frac{h_{t4}}{h_0} = 8$, $\Delta H = 43\, \text{mJ/(kg)}$, $\tau_\lambda, \text{AB} = \frac{h_{t7}}{h_{t0}} = 11 $. Write a computer program to study the effect of compression ratio on an afterburning turbojet engine. Plot the variation of TSFC, specific thrust, thermal, propulsive, and overall efficiency for compression pressure ratio ranging from 1 to 24. Compare the variation of the performance parameters with the ideal engine. Assume missing values if any.
\\
\emph{\textbf{Solution:}}
Thermal, Propulsive and Overall efficiencies are reduced by afterburning.
\\
\begin{figure}[htpb]
\includegraphics[width=0.6\textwidth]{Problem 2/TSFC.png}
\centering
\end{figure}
\begin{figure}[htpb]
\includegraphics[width=0.6\textwidth]{Problem 2/Specific thrust.png}
\centering
\end{figure}
\begin{figure}[htpb]
\includegraphics[width=0.6\textwidth]{Problem 2/Thermal eff.png}
\centering
\end{figure}
\begin{figure}[htpb]
\includegraphics[width=0.6\textwidth]{Problem 2/Propulsive eff.png}
\centering
\end{figure}
\begin{figure}[htpb]
\includegraphics[width=0.6\textwidth]{Problem 2/Overall eff.png}
\centering
\end{figure}
%2nd Problem ends

%3rd Problem starts
\item 
An ideal ramjet is to fly at an altitude of 7 km at a yet-to-be-determined Mach number. The burner exit total temperature is \SI{1778.15}{\kelvin}, and the engine will use \SI{65.7703}{\kilogram\per\second} of air. The heating value of the fuel is \SI{43015.24}{\joule\per\kilogram}. Your task is to find the Mach number at which the Thrust Specific Fuel Consumption (TSFC) is optimized, determine the optimum TSFC, calculate the thrust, and find the dimensionless thrust at this condition. To achieve this, perform a Brayton cycle analysis for the ideal ramjet, considering a specific heat ratio (\(\gamma\)) of 1.4. Vary the Mach number over the range \(0.1 < Ma < 4\) and determine the Mach number corresponding to the minimum TSFC.

Write a computer code to carry out these calculations and display the results. Plot the variation of non-dimensional thrust and TSFC with Mach number.

\emph{\textbf{Solution:}}
\\
The Mach number corresponding to the minimum TSFC is $3.4879$.
\begin{figure}[h]
\includegraphics[width=0.6\textwidth]{Problem 3/TSFC and NonF.png}
\centering
\end{figure}
%3rd Problem ends

\newpage
%4th Problem starts
\item 
An ideal turbofan with the fan exhausted operates at an altitude of \SI{4572}{\meter} at
a Mach number of 0.93. The compressor and fan pressure ratios are 17 and 2.3,
respectively. The core airflow rate is \SI{64.86}{\kilogram\per\second}, and the bypass ratio is $a$. The
fuel has a heating value of \SI{75002.5}{\joule\per\kilogram}, and the combustor exit total temperature
is \SI{2800}{\kelvin}. An afterburner is added to the core, and when lit, it results in a total
temperature of \SI{1777.78}{\kelvin} in the nozzle.
Your task is to calculate the following parameters for both the non-afterburning and afterburning cases:
Thrust (in Newtons), Dimensionless thrust, Thrust Specific Fuel Consumption (TSFC) in \si{\kilogram\per\newton\second}. Write a computer code to perform these calculations and display the results for both the non-afterburning and afterburning cases in SI units for $0.5 < \alpha < 1.5$.\\
\emph{\textbf{Solution:}}
\begin{table}[htpb]
\centering
\begin{subtable}{0.5\textwidth}
\begin{tabular}{| c | c | c | c |}
\hline
$\alpha$ & TSFC $(kg/s.N)$ & $F$ $(N)$ & Non-Dimensional Thrust \\
\hline
0.50 & 0.0199 & 92746.2167 & 1.7750 \\
\hline
0.61 & 0.0196 & 93851.6782 & 1.7961 \\
\hline
0.72 & 0.0194 & 94955.8840 & 1.8173 \\
\hline
0.83 & 0.0192 & 96058.8211 & 1.8384 \\
\hline
0.94 & 0.0190 & 97160.4759 & 1.8595 \\
\hline
1.06 & 0.0187 & 98260.8350 & 1.8805 \\
\hline
1.17 & 0.0185 & 99359.8845 & 1.9015 \\
\hline
1.28 & 0.0183 & 100457.6104 & 1.9226 \\
\hline
1.39 & 0.0181 & 101553.9982 & 1.9435 \\
\hline
1.50 & 0.0179 & 102649.0334 & 1.9645 \\
\hline
\end{tabular}
\caption{Turbojet without Afterburner}
\end{subtable}%
\\
\begin{subtable}{0.5\textwidth}
\begin{tabular}{| c | c | c | c |}
\hline
$\alpha$ & TSFC $(kg/s.N)$ & $F$ $(N)$ & Non-Dimensional Thrust \\
\hline
0.50 & 0.0153 & 77080.5231 & 3.7506 \\
\hline
0.61 & 0.0151 & 78548.8317 & 3.8142 \\
\hline
0.72 & 0.0148 & 80017.1402 & 3.8776 \\
\hline
0.83 & 0.0145 & 81485.4487 & 3.9411 \\
\hline
0.94 & 0.0143 & 82953.7573 & 4.0044 \\
\hline
1.06 & 0.0140 & 84422.0658 & 4.0677 \\
\hline
1.17 & 0.0138 & 85890.3743 & 4.1309 \\
\hline
1.28 & 0.0135 & 87358.6829 & 4.1941 \\
\hline
1.39 & 0.0133 & 88826.9914 & 4.2572 \\
\hline
1.50 & 0.0131 & 90295.2999 & 4.3202 \\
\hline
\end{tabular}
\caption{Turbojet with Afterburner}
\end{subtable}
\end{table}
\begin{figure}
\includegraphics[width=0.6\textwidth]{Problem 4/Thrust.png}
\includegraphics[width=0.6\textwidth]{Problem 4/Non F.png}
\includegraphics[width=0.6\textwidth]{Problem 4/TSFC.png}
\centering
\end{figure}
%4th Problem ends
\newpage
%5th Problem starts
\item 
The effect of compression ratio on a mixed turbofan engine with an afterburner is needed to be analyzed for the following operating parameters: $M_o = 2.0$, $P_o = 10\, \text{kPa}$, $T_o = -45^\circ\text{C}$, $\gamma = 1.4$, $c_p = 1004\, \text{J/kgK}$, $\pi_c = 12$, $\Delta H = 43, \text{MJ/kg}$, $\tau_\lambda = 8$,$\tau_{\lambda,AB} = 11$.
Write a computer program to study the effect of compression ratio on an afterburning turbofan engine. Plot the three-dimensional variation of TSFC, specific thrust, thermal, propulsive, and overall efficiency for compression pressure ratios ranging from 1 to 24 and fan compression pressure ratios ranging from 1 to 4.\\
\\
\emph{\textbf{Solution:}}
\begin{figure}[h]
\includegraphics[width=0.4\textwidth]{Problem 5/TSFC.png}
\centering
\end{figure}
\begin{figure}[h]
\includegraphics[width=0.4\textwidth]{Problem 5/Efficiencies.png}
\centering
\end{figure}
\begin{figure}[h]
\includegraphics[width=0.4\textwidth]{Problem 5/Specific Thrust.png}
\centering
\centering
\end{figure}

%5th Problem ends
%6th Problem starts
\newpage
\item
A turboprop flies at sea level at a Mach number of 0.70. It ingests 13.61 kg/s of air.
The fuel has a heating value of 43,960 kJ/kg, and the burner total temperature
is 1389 K. The work coefficient for the propeller is 1.0079. Write a computer program to study the effect of compression ratio on a turboprop engine. Plot the variation of TSFC, specific thrust, thermal, propulsive, and overall efficiency for a compression pressure ratio range from 1 to 24.\\
\\
\emph{\textbf{Solution:}}
\begin{figure}[h]
\includegraphics[width=0.6\textwidth]{Problem 6/Problem_6_TSFC.png}
\includegraphics[width=0.6\textwidth]{Problem 6/Efficiencies.png}
\centering
\end{figure}
\end{enumerate}
\newpage
\begin{center}
{\bf Matlab codes for the problems}
\end{center}
\begin{lstlisting}[style=Matlab-editor]
%Problem 1

% Input parameters %

M_a = 0.8; %Inlet Mach
T_o4 = 1667; %exit burner temperature (K)
delH = 41.8 * 10^6;%Heating value of the fuel (j/kg)
mdotc = 81.64; %core mass flow rate (kg/s)
P_atm = 101325; %Atmospeheric pressure (Pa)
T_atm = 288.15; %Atmospheric temperature (K)
gamma = 1.4;
R = 287; %gas constant for air (j/kg.K)
cp = 1005; %Specific heat (j/kg.K)

% 1) Ramjet

a_a = sqrt(gamma * R * T_atm); % Speed of sound (m/s)
u_a = M_a * a_a; % Ramjet velocity (m/s)
T_oa = T_atm * (1 + ((gamma - 1)/2)*(M_a)^2); % Total temperature at atm (K)
P_oa = P_atm * ((1 + ((gamma - 1)/2)*(M_a)^2))^(gamma/(gamma - 1)); % Total pressure at atm (Pa)
P_o4 = P_oa;
P_o3 = P_oa;
P_o8 = P_oa;
mdotf = ((mdotc * cp) * (T_o4 - T_oa))/(delH); %fuel mass flow rate (kg/s)
f = mdotf/mdotc; %fuel-air ratio
T_o8 = T_o4;
P_8 = P_atm;
M_8 = sqrt((2/(gamma-1))*((P_o8/P_8)^((gamma-1)/(gamma)) - 1)); %exit mach
S = 'K';
T_8 = T_o8/(1 + (((gamma-1)/2) * (M_8)^2)); %(K)
%fprintf('T_8 = %f K\n', T_8)
a_8 = sqrt(gamma * R * T_8); %(m/s)
u_8 = M_8 * a_8; %(m/s)
F = mdotc * (u_8 - u_a) %Thrust (N)
TSFC = mdotf/F %(kg/s.N)
tow_B = (u_8/u_a)^2;
nonF = M_a * (sqrt(tow_B) - 1) %Non-dimensional thrust

% 2) Turbojet without afterburner pi_c = 16


a_a = sqrt(gamma * R * T_atm); % Speed of sound (m/s)
u_a = M_a * a_a; % Ramjet velocity (m/s)
T_oa = T_atm * (1 + ((gamma - 1)/2)*(M_a)^2); % Total temperature at atm (K)
P_oa = P_atm * ((1 + ((gamma - 1)/2)*(M_a)^2))^(gamma/(gamma - 1)); % Total pressure at atm (Pa)
row_a = P_atm/(R * T_atm); % Density (kg/m^3)
A_in = mdotc/(row_a * u_a); % Diffuser inlet area (m^2)
%Compressor
P_o2 = P_oa;
P_o3 = (pi_c)*(P_o2);
tow_c = (pi_c)^((gamma - 1)/(gamma));
T_o2 = T_oa;
T_o3 = (T_o2) * (tow_c);
%Burner
P_o4 = P_o3;
mdotf = ((mdotc * cp) * (T_o4 - T_o3))/(delH); %fuel mass flow rate (kg/s)
%Turbine
T_o5 = T_o4 - (T_o3 - T_o2);
T_o8 = T_o5;
P_o5 = P_o4 * (T_o5/T_o4)^(gamma/(gamma-1));
%Nozzle
P_o8 = P_o5;
P_8 = P_atm;
M_8 = sqrt((2/(gamma-1))*((P_o8/P_8)^((gamma-1)/(gamma)) - 1)); %exit mach
T_8 = T_o8/(1 + (((gamma-1)/2) * (M_8)^2)); %(K)
a_8 = sqrt(gamma * R * T_8); %(m/s)
u_8 = M_8 * a_8; %(m/s)
row_8 = P_8/(R * T_8);
A_8 = mdotc / (row_8 * u_8);
%Thrust and TSFC
F = mdotc * (u_8 - u_a) %Thrust (N)
TSFC = mdotf/F %(kg/s.N)
%Non-Dimensional Thrust
nonF = F/(a_a * mdotc)

% 3) Turbojet with afterburner

T_o6 = 2333; %Total afterburner temperature

a_a = sqrt(gamma * R * T_atm); % Speed of sound (m/s)
u_a = M_a * a_a; % Ramjet velocity (m/s)
T_oa = T_atm * (1 + ((gamma - 1)/2)*(M_a)^2); % Total temperature at atm (K)
P_oa = P_atm * ((1 + ((gamma - 1)/2)*(M_a)^2))^(gamma/(gamma - 1)); % Total pressure at atm (Pa)
row_a = P_atm/(R * T_atm); % Density (kg/m^3)
A_in = mdotc/(row_a * u_a); % Diffuser inlet area (m^2)

%Compressor
P_o2 = P_oa;
P_o3 = (pi_c)*(P_o2);
tow_c = (pi_c)^((gamma - 1)/(gamma));
T_o2 = T_oa;
T_o3 = (T_o2) * (tow_c);

%Burner
P_o4 = P_o3;
mdotf = ((mdotc * cp) * (T_o4 - T_o3))/(delH); %fuel mass flow rate (kg/s)

%Turbine
T_o5 = T_o4 - (T_o3 - T_o2);
T_o8 = T_o6;
P_o5 = P_o4 * (T_o5/T_o4)^(gamma/(gamma-1));

%Afterburner

mdotfab = ((mdotc * cp) * (T_o6 - T_o5))/(delH); %Afterburner mass flow rate (kg/s)
mdot_ft = mdotf + mdotfab;

%Nozzle
P_o8 = P_o5;
P_8 = P_atm;
M_8 = sqrt((2/(gamma-1))*((P_o8/P_8)^((gamma-1)/(gamma)) - 1)); %exit mach
T_8 = T_o8/(1 + (((gamma-1)/2) * (M_8)^2)); %(K)

a_8 = sqrt(gamma * R * T_8); %(m/s)
u_8 = M_8 * a_8; %(m/s)

row_8 = P_8/(R * T_8);
A_8 = mdotc / (row_8 * u_8);

%Thrust and TSFC
F = mdotc * (u_8 - u_a) %Thrust (N)
TSFC = mdot_ft/F %(kg/s.N)

%Percentage increase in thrust compared to turbojet without afterburner
percent_increase = (((1.1809e+05) - (8.1473e+04))/(8.1473e+04))*100

%Non-Dimensional Thrust
nonF = F/(a_a * mdotc)

% 4) Turbofan with exhausted fan no Afterburner

%Diffuser
a_a = sqrt(gamma * R * T_atm); % Speed of sound (m/s)
u_a = M_a * a_a; % Ramjet velocity (m/s)
T_oa = T_atm * (1 + ((gamma - 1)/2)*(M_a)^2); % Total temperature at atm (K)
P_oa = P_atm * ((1 + ((gamma - 1)/2)*(M_a)^2))^(gamma/(gamma - 1)); % Total pressure at atm (Pa)
row_a = P_atm/(R * T_atm); % Density (kg/m^3)
A_in = (mdotc * (1 + alpha))/(row_a * u_a); % Diffuser inlet area (m^2)

%Fan
P_o2 = P_oa;
P_o7 = (pi_f)*(P_o2);
T_o2 = T_oa;
T_o7 = (T_o2)*((pi_f)^((gamma-1)/gamma));

%Fan nozzle
P_9 = P_atm;
P_o9 = P_o7;

M_9 = sqrt((2/(gamma-1))*(((P_o9/P_9)^((gamma-1)/gamma)) - 1));

%M_9 = sqrt((2/(gamma-1))*((P_o9/P_9)^((gamma-1)/(gamma)) - 1)) %exit mach
T_o9 = T_o7;
T_9 = T_o9/(1 + (((gamma-1)/2) * (M_9)^2)); %(K)
a_9 = sqrt(gamma * R * T_9);
u_9 = M_9 * a_9;

%Compressor
P_o3 = (pi_c) * P_o2;
tow_c = (pi_c)^((gamma-1)/gamma);
T_o3 = T_o2 * tow_c;

%Burner
mdotf = ((mdotc * cp)*(T_o4 - T_o3))/(delH);
f = mdotf/mdotc;

%Turbine
T_o5 = T_o4 - (T_o3 - T_o2) - ((alpha) * (T_o7 - T_o2));
P_o4 = P_o3;
tow_t = (T_o5)/(T_o4);
tow_f = (T_o7)/(T_o2);
P_o5 = (P_o4) * ((tow_t)^((gamma)/(gamma-1)));

%Primary nozzle
P_8 = P_atm;
P_o8 = P_o5;
M_8 = sqrt((2/(gamma-1))*((P_o8/P_8)^((gamma-1)/(gamma)) - 1)); %exit mach
T_o8 = T_o5;
T_8 = T_o8/(1 + (((gamma-1)/2) * (M_8)^2)); %(K)
a_8 = sqrt(gamma * R * T_8);
u_8 = M_8 * a_8;


%Thrust and TSFC
F = ((mdotc) * (u_8 - u_a)) + ((alpha * mdotc) * (u_9 - u_a))
TSFC = mdotf/F



%Non-Dimensional Thrust
nonF = F/(a_a * mdotc * (1+alpha))

%Problem 2
%To find
%TSFC
%Specific thrust
%Thermal efficiency
%Propulsive efficiency
%Overall efficiency
%pi_c = 1 to 24
%Turbojet with afterburner
%Input parameters with afterburner

M_o = 2.0;
P_o = 10000; %Static pressure(Pa)
T_o = 228.15; %Static temperature(K)
R = 287;
gamma = 1.4;
cp = 1004; %(J/kg.K)
%pi_c (P_o3/P_o2) (To vary from 1 to 24)
T_oa = T_o * (1 + ((gamma-1)/2)*(M_o)^2);
tow_lambda = 8; %((Cpt * T_o4)/(Cpc * T_o))
delH = 43 * 10^6; %Heating value (J/kg)
tow_lambda_AB = 11; %Temperature ratio of afterburner (T_o7/T_o)
a_o = sqrt(gamma * R * T_o);
V_o = M_o * a_o;
T_t4 = (tow_lambda)*T_o; %Burner exit temperature
T_t7 = (tow_lambda_AB)*(T_o); %Total temperature at exit of afterburner
P_to = P_o*((1+((gamma-1)/2)*(M_o)^2)^((gamma)/(gamma-1)));
T_to = T_o*(1+((gamma-1)/2)*(M_o)^2);
T_t2 = T_to;
P_t2 = P_to;
tow_r = 1 + (((gamma-1)/2)*(M_o)^2);
y = 1;

for pi_c = (1:24)
P_t3(1,y) = P_t2 * pi_c;
tow_c(1,y) = (pi_c)^((gamma-1)/gamma); %Compressor temperature ratio
T_t3 = T_t2 .* tow_c;
P_t4 = P_t3;
f = (cp .* (T_t4 - T_t3))./(delH - (cp * T_t4));
T_t5 = T_t4 - ((T_t3 - T_t2)/(1+f));
pi_t = (T_t5/T_t4)^((gamma)/(gamma-1));
P_t5 = P_t4 * pi_t;
P_t7 = P_t5;
f_ab = ((1+f)*((cp * T_t7)-(cp * T_t5)))/((delH - (cp * T_t7)));
P_t9 = P_t7;
T_t9 = T_t7;
P_9 = P_o;
M_9 = sqrt((2/(gamma-1))*(((P_t9/P_9).^((gamma-1)/(gamma))) - 1));
T_9 = T_t9./(1 + ((gamma-1)/2)*((M_9).^2));
a_9 = sqrt(gamma * R * T_9);
V_9 = M_9.*a_9;
specific_thrust = ((1+f+f_ab).*V_9) - V_o;
non_specific_thrust = ((1+f+f_ab).*(V_9/a_o)) - M_o;
TSFC = (f+f_ab)./(specific_thrust);
A = ((1+f+f_ab).*((V_9).^2)/2) - ((V_o).^2/2);
B = (f + f_ab).* delH;
Thermal_eff = (A./B);
Propulsive_eff = ((specific_thrust * V_o)./A);
Overall_eff = Thermal_eff .* Propulsive_eff;
y = y+1;
end

x = (1:24);
non_specific_thrust;
TSFC;
Thermal_eff;
Propulsive_eff;
Overall_eff;

%Problem 2
%To find
%TSFC
%Specific thrust
%Thermal efficiency
%Propulsive efficiency
%Overall efficiency
%pi_c = 1 to 24

%Input parameters
%Turbojet without afterburner

M_o = 2.0;
P_o = 10000; %Static pressure(Pa)
T_o = 228.15; %Static temperature(K)
R = 287;
gamma = 1.4;
cp = 1004; %(J/kg.K)
%pi_c (P_o3/P_o2) (To vary from 1 to 24)
T_oa = T_o * (1 + ((gamma-1)/2)*(M_o)^2);
tow_lambda = 8; %((Cpt * T_o4)/(Cpc * T_o))
delH = 43 * 10^6; %Heating value (J/kg)
tow_lambda_AB = 11; %Temperature ratio of afterburner (T_o7/T_o)
a_o = sqrt(gamma * R * T_o);
V_o = M_o * a_o;
T_t4 = (tow_lambda)*T_o; %Burner exit temperature
T_t7 = (tow_lambda_AB)*(T_oa); %Total temperature at exit of afterburner
P_to = P_o*((1+((gamma-1)/2)*(M_o)^2)^((gamma)/(gamma-1)));
T_to = T_o*(1+((gamma-1)/2)*(M_o)^2);
T_t2 = T_to;
P_t2 = P_to;
tow_r = 1 + (((gamma-1)/2)*(M_o)^2);
y = 1;

for pi_c = (1:24)
P_t3(1,y) = P_t2 * pi_c;
tow_c(1,y) = (pi_c)^((gamma-1)/gamma); %Compressor temperature ratio
T_t3 = T_t2 .* tow_c;
P_t4 = P_t3;
f = (cp .* (T_t4 - T_t3))./(delH - (cp * T_t4));
T_t5 = T_t4 - ((T_t3 - T_t2)/(1+f));
pi_t = (T_t5/T_t4)^((gamma)/(gamma-1));
P_t5 = P_t4 * pi_t;
%P_t7 = P_t5;
%f_ab = ((1+f)*((cp * T_t7)-(cp * T_t5)))/((delH - (cp * T_t7)));
P_t9 = P_t5;
T_t9 = T_t5;
P_9 = P_o;
M_9 = sqrt((2/(gamma-1))*(((P_t9/P_9).^((gamma-1)/(gamma))) - 1));
T_9 = T_t9./(1 + ((gamma-1)/2)*((M_9).^2));
a_9 = sqrt(gamma * R * T_9);
V_9 = M_9.*a_9;
specific_thrust_i = ((1+f).*V_9) - V_o;
non_specific_thrust_i = ((1+f).*(V_9/a_o)) - M_o;
TSFC_i = (f)./(specific_thrust_i);
A = ((1+f).*((V_9).^2)/2) - ((V_o).^2/2);
B = (f).* delH;
Thermal_eff_i = (A./B);
%Propulsive_eff_i = ((specific_thrust_i * V_o)./A);
%Thermal_eff_i = 1 - (T_o./T_t3);
Propulsive_eff_i = 2./(1 + (V_9/V_o));
Overall_eff_i = Thermal_eff_i .* Propulsive_eff_i;
y = y+1;
end

x = (1:24);
specific_thrust_i;
TSFC_i;
Thermal_eff_i;
Propulsive_eff_i;
Overall_eff_i;

plot(x,TSFC,'LineWidth',2)
hold on 
plot(x,TSFC_i,'LineWidth',2)
xlabel('Compression pressure ratio');
ylabel('TSFC (kg/sN)');
title('TSFC vs. Compression pressure ratio');
legend('Turbojet with afterburner','Turbojet without afterburner')
grid on
hold off

plot(x,specific_thrust,'LineWidth',2)
hold on 
plot(x,specific_thrust_i,'LineWidth',2)
xlabel('Compression pressure ratio');
ylabel('Specific thrust');
title('Specific thrust vs. Compression pressure ratio');
legend('Turbojet with afterburner','Turbojet without afterburner');
grid on
hold off

plot(x,Thermal_eff,'LineWidth',2)
hold on 
plot(x,Thermal_eff_i,'LineWidth',2)
xlabel('Compression pressure ratio');
ylabel('Thermal efficiency');
title('Thermal efficiency vs. Compression pressure ratio');
legend('Turbojet with afterburner','Turbojet without afterburner')
grid on
hold off

plot(x,Propulsive_eff,'LineWidth',2)
hold on 
plot(x,Propulsive_eff_i,'LineWidth',2)
xlabel('Compression pressure ratio');
ylabel('Propulsive efficiency');
title('Propulsive efficiency vs. Compression pressure ratio');
legend('Turbojet with afterburner','Turbojet without afterburner')
grid on
hold off

plot(x,Overall_eff,'LineWidth',2)
hold on 
plot(x,Overall_eff_i,'LineWidth',2)
xlabel('Compression pressure ratio');
ylabel('Overall efficiency');
title('Overall efficiency vs. Compression pressure ratio');
legend('Turbojet with afterburner','Turbojet without afterburner')
grid on
hold off

%Problem 3

M_a = linspace(0.1,4,100); %Mach number range - To find
T_o4 = 1778.15; %exit burner temperature (K)
delH = 43015.24; %Heating value of the fuel (j/kg)
mdotc = 65.7703; %core mass flow rate (kg/s)
P_atm = 4.11 * 10^4; %Atmospeheric pressure (Pa)
T_atm = 242.65; %Atmospheric temperature (K)
gamma = 1.4;
R = 287; %gas constant for air (j/kg.K)
cp = 1005; %Specific heat (j/kg.K)
a_a = sqrt(gamma * R * T_atm);% Speed of sound (m/s)
y = 1;
for x = linspace(0.1,4,100)
    u_a(1,y) = x.*a_a;
    T_oa(1,y) = T_atm * (1 + ((gamma - 1)/2)*(x).^2);
    T_o3 = T_oa;
    P_oa(1,y) = P_atm * ((1 + ((gamma - 1)/2).*(x).^2)).^(gamma/(gamma - 1)); % Total pressure at atm (Pa)
    P_o4 = P_oa;
    P_o3 = P_oa;
    P_o8 = P_oa;
    mdotf = ((mdotc * cp) * (T_o4 - T_oa))/(delH);
    f = mdotf/mdotc;
    T_o8 = T_o4;
    P_8 = P_atm;
    M_8 = sqrt((2/(gamma-1))*((P_o8/P_8).^((gamma-1)/(gamma)) - 1)); %exit mach
    T_8 = T_o8./(1 + (((gamma-1)/2) * (M_8).^2)); %(K)
    a_8 = sqrt(gamma * R .* T_8);
    u_8 = M_8 .* a_8;%(m/s)
    F = mdotc .* (u_8 - u_a);%Thrust (N)
    TSFC = mdotf./F; %(kg/s.N)
    tow_B = (u_8./u_a).^2;
    nonF = F./(mdotc * a_a); %Non-dimensional thrust
    y = y+1;
end

[min_tsfc, min_tsfc_index] = min(TSFC);
optimal_mach = M_a(min_tsfc_index)
TSFC;
nonF;

yyaxis left;
plot(M_a, TSFC, 'b','LineWidth', 2);
xlabel('Mach Number');
ylabel('TSFC (kg/sN)');
title('Non-Dimensional Thrust and TSFC vs. Mach Number');

yyaxis right;
plot(M_a,nonF, 'r','LineWidth', 2);
ylabel('Non-Dimensional Thrust');
grid on
legend('TSFC', 'Non-Dimensional Thrust');

%Problem 4
%ToFind
%Thrust
%Dimensionless Thrust
%TSFC

%Turbofan without afterburner

% Input parameters

%Sea level conditions
T_sea = 288.16;
P_sea = 101325;
row_sea = 1.225;
H = 4572; %Given altitude
g = 9.81;
R = 287;
a = -0.0065;
H1 = 0;
cp = 1005;
T_a = T_sea + (a * (H - H1));
P_a = P_sea * ((T_a/T_sea)^(-g/(a*R)));
row_a = row_sea * ((T_a/T_sea)^(-(g/(a*R))-1));

%Given parameters
M_a = 0.93;
pi_c = 17;
pi_f = 2.3;
mdotc = 64.86;
%alpha = 0.5; %Varies from 0.5 to 1.5
delH = 75002.5;
T_o4 = 2800; %Combustor exit total temperature
gamma = 1.4;
y = 1;
for alpha = linspace(0.5,1.5,10)
%Diffuser
a_a = sqrt(gamma * R * T_a); % Speed of sound (m/s)
u_a = M_a * a_a; % Ramjet velocity (m/s)
T_oa = T_a * (1 + ((gamma - 1)/2)*(M_a)^2); % Total temperature at atm (K)
P_oa = P_a * ((1 + ((gamma - 1)/2)*(M_a)^2))^(gamma/(gamma - 1)); % Total pressure at atm (Pa)
row_a = P_a/(R * T_a); % Density (kg/m^3)
%A_in = (mdotc * (1 + alpha))/(row_a * u_a) % Diffuser inlet area (m^2)
P_o2 = P_oa;
T_o2 = T_oa;
P_o3 = (pi_c) * P_o2;
P_o4 = P_o3;
%Fan
P_o7 = (pi_f) * (P_o2);
T_o7 = T_o2 * (pi_f)^((gamma-1)/gamma);
%Fan nozzle
P_9 = P_a;
P_o9 = P_o7;
M_9 = sqrt((2/(gamma-1))*((P_o9/P_9)^((gamma-1)/(gamma)) - 1)); %exit mach of fan nozzle
T_o9 = T_o7;
T_9 = T_o9/(1 + (((gamma-1)/2) * (M_9)^2)); %(K)
a_9 = sqrt(gamma * R * T_9);
u_9 = M_9 * a_9;
%Turbine
tow_c = (pi_c)^((gamma - 1)/(gamma));
T_o3 = (T_o2) * (tow_c);
T_o5 = T_o4 - (T_o3 - T_o2) - ((alpha) .* (T_o7 - T_o2));
tow_t = (T_o5)/(T_o4);
tow_f = (T_o7)/(T_o2);
P_o5 = (P_o4) * ((tow_t)^((gamma)/(gamma-1)));
%Primary nozzle
P_8 = P_a;
P_o8 = P_o5;
M_8 = sqrt((2/(gamma-1))*((P_o8/P_8)^((gamma-1)/(gamma)) - 1)); %exit mach of nozzle
T_o8 = T_o5;
T_8 = T_o8/(1 + (((gamma-1)/2) * (M_8)^2)); %(K)
a_8 = sqrt(gamma * R * T_8);
u_8 = M_8 * a_8;
%Thrust and TSFC
F(1,y) = ((mdotc) * (u_8 - u_a)) + ((alpha .* mdotc) * (u_9 - u_a));
mdotf = ((mdotc * cp) * (T_o4 - T_o3))/(delH); %fuel mass flow rate (kg/s)
TSFC = mdotf./F;
%Non-Dimensional Thrust
nonF = F/(a_a * mdotc * (1+alpha));
y = y+1;
end

x = linspace(0.5,1.5,10);
F;
TSFC;
nonF;
Table_wab = table([x; TSFC; F; nonF])
writetable(Table_wab, 'table_withoutab.csv', 'Delimiter',',')

plot(x,TSFC,'LineWidth',2);
xlabel('Bypass ratio');
ylabel('TSFC (kg/sN)');
title('TSFC vs. Bypass ratio');
grid on
plot(x,nonF,'LineWidth',2);
xlabel('Bypass ratio');
ylabel('Non-Dimensional Thrust');
title('Non-Dimensional Thrust and TSFC vs. Bypass ratio');
grid on
plot(x,F,'LineWidth',2);
xlabel('Bypass ratio');
ylabel('Thrust (N)');
title('Thrust vs. Bypass ratio');
grid on


%Turbofan with afterburner

% Input parameters

%Sea level conditions
T_sea = 288.16;
P_sea = 101325;
row_sea = 1.225;
H = 4572; %Given altitude
g = 9.81;
R = 287;
a = -0.0065;
H1 = 0;
cp = 1005;
T_a = T_sea + (a * (H - H1));
P_a = P_sea * ((T_a/T_sea)^(-g/(a*R)));
row_a = row_sea * ((T_a/T_sea)^(-(g/(a*R))-1));

%Given parameters
M_a = 0.93;
pi_c = 17;
pi_f = 2.3;
mdotc = 64.86;
%alpha = 0.5; %Varies from 0.5 to 1.5
delH = 75002.5;
T_o4 = 2800; %Combustor exit total temperature
gamma = 1.4;
T_o6 = 1777.78; %A/B exit total temperature
y = 1;
for alpha = linspace(0.5,1.5,10)
%Diffuser
a_a = sqrt(gamma * R * T_a); % Speed of sound (m/s)
u_a = M_a * a_a; % Ramjet velocity (m/s)
T_oa = T_a * (1 + ((gamma - 1)/2)*(M_a)^2); % Total temperature at atm (K)
P_oa = P_a * ((1 + ((gamma - 1)/2)*(M_a)^2))^(gamma/(gamma - 1)); % Total pressure at atm (Pa)
row_a = P_a/(R * T_a); % Density (kg/m^3)
%A_in = (mdotc * (1 + alpha))/(row_a * u_a) % Diffuser inlet area (m^2)
P_o2 = P_oa;
T_o2 = T_oa;
P_o3 = (pi_c) * P_o2;
P_o4 = P_o3;
%Fan
P_o7 = (pi_f) * (P_o2);
T_o7 = T_o2 * (pi_f)^((gamma-1)/gamma);
%Fan nozzle
P_9 = P_a;
P_o9 = P_o7;
M_9 = sqrt((2/(gamma-1))*((P_o9/P_9)^((gamma-1)/(gamma)) - 1)); %exit mach of fan nozzle
T_o9 = T_o7;
T_9 = T_o9/(1 + (((gamma-1)/2) * (M_9)^2)); %(K)
a_9 = sqrt(gamma * R * T_9);
u_9 = M_9 * a_9;
%Turbine
tow_c = (pi_c)^((gamma - 1)/(gamma));
T_o3 = (T_o2) * (tow_c);
T_o5 = T_o4 - (T_o3 - T_o2) - ((alpha) .* (T_o7 - T_o2));
tow_t = (T_o5)/(T_o4);
tow_f = (T_o7)/(T_o2);
P_o5 = (P_o4) * ((tow_t)^((gamma)/(gamma-1)));
%Afterburner
mdotf = ((mdotc * cp) * (T_o4 - T_o3))/(delH); %fuel mass flow rate (kg/s)
mdotfab = ((mdotc * cp) * (T_o6 - T_o5))/(delH); %A/B fuel mass flow rate (kg/s)
mdotft = ((mdotc * cp * T_a)/(delH))*((T_o6/T_oa) + (alpha * tow_f) - (1 + alpha));
%Nozzle
T_o8 = T_o6;
T_8 = (T_o8)./(1 + ((gamma-1)/2)*(M_8).^2);
a_8 = sqrt(gamma * R * T_8);
u_8 = M_8 .* a_8;
u8byu6 = sqrt((T_o6/T_a)*(1 - (1/((T_oa/T_a)*tow_c*tow_t)))/((T_oa/T_a)-1));
%Thrust
mdots = alpha * mdotc;
F_ab(1,y) = ((mdotc * u_a)*((u_8/u_a)-1)) + ((mdots * u_a)*((u_9/u_a)-1));
TSFC_ab = mdotft./F_ab;
A = (T_o6/T_a)/((T_oa/T_a) - 1);
B = ((T_oa/T_a)*(tow_c))*(1 - (((T_oa/T_a) * (T_a/T_o4))*(((tow_c)-1) + ((alpha) * (tow_f - 1)))));
C = sqrt((((T_oa/T_a)*(tow_f)) - 1)/((T_oa/T_a) - 1)) - 1;
nonThrust_ab(1,y) = (M_a * (sqrt(A * (1 - (1/B))) - 1)) + ((alpha * M_a) * C);
y = y+1;
end
nonThrust_ab

plot(x,TSFC,'LineWidth',2);
hold on
plot(x,TSFC_ab,'LineWidth',2);
xlabel('Bypass ratio');
ylabel('TSFC (kg/sN)');
title('TSFC vs. Bypass ratio');
grid on
legend('TSFC without afterburnet','TSFC with afterburner')
hold off

plot(x,nonF,'LineWidth',2);
hold on
plot(x,nonThrust_ab,'LineWidth',2);
xlabel('Bypass ratio');
ylabel('Non-Dimensional Thrust');
title('Non-Dimensional Thrust and TSFC vs. Bypass ratio');
grid on
legend('Non-Dimensional Thrust without afterburnet','Non-Dimensional Thrust with afterburner')
hold off

plot(x,F,'LineWidth',2);
hold on
plot(x,F_ab,'LineWidth',2);
xlabel('Bypass ratio');
ylabel('Thrust (N)');
title('Thrust vs. Bypass ratio');
grid on
hold off
legend('Thrust without afterburnet','Thrust with afterburner')

%Problem 5
%Mixed Turbofan with afterburner

%Input parameters
M_a = 2.0;
P_a = 10000; %Static pressure(Pa)
T_a = 228.15; %Static temperature(K)
R = 287;
gamma = 1.4;
cp = 1004; %(J/kg.K)
%pi_c = 12; %(P_o3/P_o2) (To vary from 1 to 24)
%pi_f (P_o13/P_o2) To vary from 1 to 4
tow_lambda = 8; %((Cpt * T_o4)/(Cpc * T_a))
delH = 43 * 10^6; %Heating value (J/kg)
tow_lambda_AB = 11; %Temperature ratio of afterburner (T_o6/T_a)
P_oa = P_a * (1 + ((gamma-1)/2)*((M_a)^2))^((gamma)/(gamma-1));
T_oa = T_a * (1 + ((gamma-1)/2)*((M_a)^2));
pi_r = P_oa/P_a;
a_a = sqrt(gamma * R * T_a);
u_a = a_a * M_a;
%T_o4 = (tow_lambda)*T_a; %Burner exit temperature
%T_o6 = (tow_lambda_AB)*(T_oa); %Total temperature at exit of afterburner
alpha = 1.2;
pi_c = (1:1:24);
pi_f = (1:1:4);
mdotc = 74; %primary flow rate kg/s

for i = 1:1:4
    for j = 1:1:24
        %Fan
        P_o2(i,j) = pi_f(i)*P_oa;
        T_o2(i,j) = T_oa*((pi_f(i))^((gamma-1)/(gamma)));
        tow_f(i,j) = T_o2(i,j)/T_oa;
        %Compressor
        P_o3(i,j) = pi_c(j)*P_o2(i,j);
        T_o3(i,j) = T_o2(i,j)*(pi_c(j))^((gamma-1)/(gamma));
        %Burner
        P_o4(i,j) = P_o3(i,j);
        T_o4(i,j) = 8 * T_oa;
        tow_b(i,j) = T_o4(i,j)/T_o3(i,j);
        %Turbine
        T_o5(i,j) = T_o4(i,j) + T_o2(i,j) - T_o3(i,j) - alpha*(T_o2(i,j)-T_oa);
        P_o5(i,j) = P_o4(i,j) * (T_o5(i,j)/T_o4(i,j))^(gamma/(gamma-1));
        %Bypass Duct
        T_o7(i,j) = T_o2(i,j);
        P_o7(i,j) = P_o2(i,j);
        %Mixer
        T_o55(i,j) = T_oa*tow_f(i,j)*((alpha+tow_b(i,j)/(alpha+1)));
        P_o55(i,j) = P_o5(i,j);
        %Afterburner
        T_o6(i,j) = 11 * T_a;
        P_o6(i,j) = P_o55(i,j);
        %Nozzle
        M_8(i,j) = sqrt((2/(gamma-1))*(((P_o55(i,j)/P_a)^((gamma-1)/gamma))-1));
        T_8(i,j) = T_o6(i,j)/(1 + ((gamma-1)/2)*M_8(i,j)^2);
        u_8(i,j) = M_8(i,j)*sqrt(gamma*R*T_8(i,j));
        %Specific_thrust
        Specific_thrust(i,j) = (1+alpha)*(u_8(i,j) - u_a);
        %TSFC
        mdotf(i,j) = (mdotc * cp *(T_o4(i,j) - T_o3(i,j)))/(delH);
        mdotf_ab(i,j) = (1+alpha)*mdotc*cp*(T_o4(i,j) - T_o3(i,j))/(delH);
        mdot_tot(i,j) = mdotf(i,j) + mdotf_ab(i,j);
        TSFC = mdot_tot/(Specific_thrust(i,j) * mdotc);
        %Thermal efficiency
        Thermal_eff(i,j) = (Specific_thrust(i,j)*mdotc*u_a)/(mdot_tot(i,j)*delH);
        %Propulsive efficiency
        Prop_eff(i,j) = 2*u_a/(u_a + u_8(i,j));
        %Overall efficiency
        Overall_eff(i,j) = Thermal_eff(i,j)*Prop_eff(i,j);
    end
end

surf(TSFC)
xlabel('\pi_c')
ylabel('\pi_f')
zlabel('TSFC')
grid on

surf(Specific_thrust)
xlabel('\pi_c')
ylabel('\pi_f')
zlabel('Specific thrust')
grid on


surf(pi_c,pi_f, Prop_eff)
hold on
surf(pi_c,pi_f, Thermal_eff)
hold on
surf(pi_c,pi_f, Overall_eff)
xlabel('\pi_c')
ylabel('\pi_f')
zlabel('Efficiency')
%zlabel('propulsive efficiency')
grid on

% Problem 6
% Turbo Prop
% To find
% TSFC
% Specific thrust
% Thermal eff
% Propulsive eff
% Overall eff
% Compression pressure ratio - 1 to 24

% Given values
M_a = 0.70;
mdotc = 13.61;
delH = 43960000;
T_o4 = 1389;
C_w_p = 1.0079; % Work coefficient
P_a = 101325;
P_8 = 101325;
T_a = 288.15;
gamma = 1.4;
R = 287;
cp = 1005;
pi_c = linspace(1,24,100);
y = 1;

for i = 1:length(pi_c)
    %Diffuser
    u_a = M_a * sqrt(gamma * R * T_a);
    T_oa = T_a * (1 + ((gamma-1)/2)*M_a^2);
    T_o2 = T_oa;
    P_oa = P_a * (1 + ((gamma-1)/2)*M_a^2)^(gamma/(gamma-1));
    %Compressor
    P_o2 = P_oa;
    P_o3(1,y) = (pi_c(i)) * (P_o2);
    tow_c = (pi_c(i))^((gamma-1)/gamma);
    T_o3(1,y) = tow_c * T_o2;
    %Burner
    mdotf(1,y) = (mdotc * cp * (T_o4 - T_o3(1,y)))/(delH);
    P_o4(1,y) = P_o3(1,y);
    f(1,y) = mdotf(1,y)/mdotc;
    %Turbine
    T_o5(1,y) = T_o4 - (T_o3(1,y) - T_o2); %- (C_w_p * T_a);
    tow_t = T_o5(1,y)/T_o4;
    P_o5(1,y) = P_o4(1,y) * (tow_t)^(gamma/(gamma-1));
    %Nozzle
    P_o8(1,y) = P_o5(1,y);
    P_8 = P_a;
    M_8(1,y) = sqrt((2/(gamma-1))*(((P_o8(1,y)/P_8)^((gamma-1)/(gamma))) - 1));
    T_o8(1,y) = T_o5(1,y);
    T_8(1,y) = T_o8(1,y)/(1 + (((gamma-1)/2)*(M_8(1,y))^2));
    u_8(1,y) = M_8(1,y) * sqrt(gamma * R * T_8(1,y));
    %Propeller
    tow_5 = ((T_oa / T_a) * (tow_c) * (tow_t));
    A = sqrt(((T_o4 / T_a) * (T_a / T_oa) * ((tow_5 - 1) / (tow_c))) / ((T_oa / T_a) - 1)) - 1;
    C_w_j = (gamma - 1) * (M_a)^2 * A;
    C_w_e = C_w_p + C_w_j;
    %Thrust Power
    Thrust_Power(1,y) = C_w_e * mdotc * cp * T_a;
    %Thrust
    F_t(1,y) = Thrust_Power(1,y)/u_a;
    %SFC
    SFC(1,y) = mdotf(1,y)/Thrust_Power(1,y);
    %TSFC
    TSFC(1,y) = mdotf(1,y)/F_t(1,y);
    %Thermal efficiency
    mdot_e = mdotc + mdotf;
    Thermal_eff(1,y) = (((mdot_e/2)*(u_8(1,y))^2) - ((mdotc/2)*(u_a)^2))/(mdotf * delH);
    %Propulsive efficiency
    Propulsive_eff(1,y) = 2/(1 + (u_8(1,y)/u_a));
    %Overall efficiency
    Overall_eff(1,y) = Thermal_eff(1,y)*Propulsive_eff(1,y);
    y = y+1;
end
Thermal_eff;
Propulsive_eff;
Overall_eff;

plot(pi_c,TSFC,'LineWidth',2)
xlabel('Compression Pressure Ratio (\pi_c)')
ylabel('TSFC (kg/s.N)')
title('TSFC vs \pi_c')
grid on

plot(pi_c,Thermal_eff, 'LineWidth',2)
%xlabel('Compression Pressure Ratio (\pi_c)')
hold on
plot(pi_c,Propulsive_eff, 'LineWidth',2)
%xlabel('Compression Pressure Ratio (\pi_c)')
hold on
plot(pi_c,Overall_eff, 'LineWidth',2)
xlabel('Compression Pressure Ratio (\pi_c)')
ylabel('Efficiency')
grid on
legend('Thermal efficiency','Propulsive efficiency','Overall efficiency')
\end{lstlisting}

\begin{center}
{\bf References}
\end{center}
\begin{enumerate}
    \item Flack, R. (2005). Fundamentals of Jet Propulsion with Applications (Cambridge Aerospace Series). Cambridge: Cambridge University Press. doi:10.1017/CBO9780511807138
    \item Mattingly,Jack D..Elements of Gas Turbine Propulsion.United States,American Institute of Aeronautics and Astronautics,2005.
    \item Farokhi,Saeed.Aircraft Propulsion.Germany,Wiley,2014.
    \item Lec-31 Fundamentals of Aerospace Propulsion, \href{https://youtu.be/x7jw5cJWydY}.
\end{enumerate}

\end{document}
